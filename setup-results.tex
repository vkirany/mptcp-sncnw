% ###########################################################################
\section{Experimental Setup and Results}
\label{sec:evaluation_presentation}
% ###########################################################################
This section describes the experimental setup and presents the results comparing
the Linux implementations of TCP and MPTCP for transporting web traffic. We
evaluated the protocols performance through emulations, identifying the impact of various network
parameters in both homogeneous and heterogeneous scenarios. 

\begin{figure*}
  \centering
  \begin{tabular}{ccc}
  \subfloat[Wikipedia download, $15$ objects, $72$Kb total size.\label{fig:web:wikipedia}]{
   \includegraphics[width=.28\linewidth]{plots/Wikipedia_avgDelay_bar}
  } &
  \subfloat[Amazon download, $54$ objects, $1$Mb total size.\label{fig:web:amazon}]
  {
    \includegraphics[width=.28\linewidth]{plots/Amazon_avgDelay_bar}
  } &
  \subfloat[Huffington Post download, $134$ objects, $3.9$Mb total size.\label{fig:web:huffpost}]
  {
    \includegraphics[width=.28\linewidth]{plots/Huffpost_avgDelay_bar}
  } 
%  \\
%  \subfloat[Wikipedia download, XYZ objects of average size XYZKB.\label{fig:web:wikipediawbg}]{
%	  \includegraphics[width=.32\linewidth]{plots/wikipedia_avgDelayWBG}
%  } &
%  \subfloat[Amazon download, XYZ objects of average size XYZKB.\label{fig:web:amazonwbg}]
%  {
%    \includegraphics[width=.32\linewidth]{plots/Amazon_avgDelayWBG}
%  } &
%  \subfloat[Huffington Post download, XYZ objects of average size XYZKB.\label{fig:web:huffpostwbg}]
%  {
%    \includegraphics[width=.32\linewidth]{plots/huffpost_avgDelayWBG}
%  }
  \end{tabular}
  \caption{Average web transfer delay, with standard deviation.}
  \label{fig:web}
\end{figure*}


\begin{figure*}
  \centering
  \begin{tabular}{ccc}
  \subfloat[Wikipedia download, $15$ objects, $72$Kb total size.\label{fig:web:wikipedia-bg}]{
   \includegraphics[width=.28\linewidth]{plots/wikipedia_avgDelayWBG}
  } &
  \subfloat[Amazon download, $54$ objects, $1$Mb total size.\label{fig:web:amazon-bg}]
  {
    \includegraphics[width=.28\linewidth]{plots/Amazon_avgDelayWBG}
  } &
  \subfloat[Huffington Post download, $134$ objects, $3.9$Mb total size.\label{fig:web:huffpost-bg}]
  {
    \includegraphics[width=.28\linewidth]{plots/huffpost_avgDelayWBG}
  }
 \end{tabular}
  \caption{With background flows: Average web transfer delay, with standard deviation.}
  \label{fig:web-bg}
\end{figure*}

For the emulations, we use the Common Open Research Emulator (CORE)~\cite{CORE}. The setup consists of 
two end-hosts, one acting as a mobile device equipped with two interfaces and the other as a web server. To reach the
server, each interface of the mobile client is connected to an (emulated) access network of its own. 

To evaluate the effect of having both homogeneous and heterogeneous network access we performed 3 sets 
of experiments where the interfaces used a combination of WLAN and 3G access (i.e., WLAN-WLAN, WLAN-3G, 3G-3G). 
The capacity of the WLAN was randomly chosen in the range 20-30Mbps, whereas for 3G we use values in the range 3-5Mbps. 
The propagation delays are choosen in the range of 20-25ms and 65-75ms for WLAN and 3G, respectively.

For each experiment, the server stores a small set of files from three different classes
of websites: Wikipedia, Amazon and Huffpost. Each website contains different number of objects of different sizes. 
Of the three, Wikipedia was the smallest ($15$ objects, $72$Kb total size) followed by Amazon ($54$ objects, $1$Mb total size) 
and then Huffpost ($138$ objects, $3.9$Mb total size). The sites were then requested and downloaded from the mobile 
client using six concurrent connections each using TCP or MPTCP, in respective experiments.

We define the user experience of web browsing as the time needed to download a web page, hence, we measure the average flow completion time for each class of web page.

There is a significant impact of the congestion level on the behavior of congestion control protocols~\cite{ha-background-traffic-comnet-2007}. We therefore conduct
experiments both without and with background traffic.  The background traffic is a mix of TCP and UDP flows constituting one long TCP flow and 4 UDP flows. The
aggregate usage of background flows maintained at 10\% of the bottleneck link capacity to be more realistic. The UDP flows carry data at 500kbps each in the
WLAN-WLAN scenario and 100kbps each in the 3G-3G scenario.  In each run, the background flows start before the foreground experimental traffic and end after
the experimental traffic.


Figure~\ref{fig:web} shows the average transfer times for web traffic, with standard variation over $30$ repetitions for 
each configuration without background traffic. Each figure depicts the average transfer time when using MPTCP, 
TCP on one of the interfaces (TCP1, TCP2), or TCP on the best available interface (MinTCP); the results are grouped according
to the emulated access networks used (WLAN-WLAN/3G-3G/WLAN-3G). We observe that MPTCP cannot reduce transfer times for the Wikipedia site, 
but is able to do so for both Amazon and Huffpost (especially in the 3G-3G scenarios). The reason for the poor performance of retrieving 
the Wikipedia site is simply that the amount of data is so small that it can be transmitted within TCP's initial window (given the six concurrent connections used).
Employing more paths in such scenarios is not useful as long as the path itself can sustain the traffic load. However, for the other sites the amount of data is
much larger and the transfer time can be reduced by MPTCP's implicit load-balancing over its available subflows; the positive effects of this load-balancing peek
when the path characteristics of the subflows are homogeneous, as possible head-of-line blocking effects are less prevalent.


Figure \ref{fig:web-bg} illustrates the average transfer times for web traffic for emulations conducted with background traffic. 


For Wikipedia, with background traffic the results show similar behavior in the symmetric scenarios (WLAN-WLAN,3G-3G) as that of the 
no background case. In the asymmetric scenario (WLAN-3G) the performance of MPTCP is not significantly lower than that of the TCP as
seen in no background case. The reason could be the random losses WLAN. However a deeper study on the percentage of packets 
transfered on each of the paths is required.

\begin{figure*}
  \centering
  \begin{tabular}{ccc}
  \subfloat[WLAN-WLAN Scenario packet path split.\label{fig:web:pshare-wlan-wlan}]{
   \includegraphics[width=.28\linewidth]{plots/pshare-WLAN-WLAN}
  } &
  \subfloat[3G-3G Scenario packet path split. \label{fig:web:pshare-3g-3g}]
  {
    \includegraphics[width=.28\linewidth]{plots/pshare-3G-3G}
  } &
  \subfloat[WLAN-3G Scenario packet path split. \label{fig:web:pshare-wlan-3g}]
  {
    \includegraphics[width=.28\linewidth]{plots/pshare-WLAN-3G}
  }
 \end{tabular}
  \caption{MPTCP percent packets on each path without background traffic}
  \label{fig:web-pshare}
\end{figure*}

\begin{figure*}
  \centering
  \begin{tabular}{ccc}
  \subfloat[WLAN-WLAN Scenario packet path split.\label{fig:web:pshare-wlan-wlan-bg}]{
   \includegraphics[width=.28\linewidth]{plots/pshare-WLAN-WLAN-BG}
  } &
  \subfloat[3G-3G Scenario packet path split. \label{fig:web:pshare-3g-3g-bg}]
  {
    \includegraphics[width=.28\linewidth]{plots/pshare-3G-3G-BG}
  } &
  \subfloat[WLAN-3G Scenario packet path split. \label{fig:web:pshare-wlan-3g-bg}]
  {
    \includegraphics[width=.28\linewidth]{plots/pshare-WLAN-3G-BG}
  }
 \end{tabular}
  \caption{MPTCP percent packets on each path with background traffic}
  \label{fig:web-pshare-bg}
\end{figure*}

For Amazon and Huffpost cases, we observe similar behavior in all the three scenairos. As expected, the download times are higher in the background case.

\begin{figure}[!th]
\begin{minipage}[t]{0.48\textwidth}
\begin{center}
\includegraphics[width=.98\linewidth]{plots/MPTCP-Web}
\end{center}
\caption{Comparison of web traffic latency using MPTCP and TCP}
  \label{fig:web-summary}
\end{minipage}
\end{figure}

Further we analyze the percent share of traffic on each paths for all scenarios. This helps in understanding the improvement or degradation of performance
in the usage of MPTCP in all scenarios.



In Figure \ref{fig:web-summary} we provide a systematic comparison of web traffic latency in using MPTCP and TCP across all scenarios.
