% ###########################################################################
\section{Introduction}
\label{sec:introduction}
% ###########################################################################
In the recent days, there has been significant growth in the use of mobile devices for data access especially access to the Internet.
A large portion of these mobile devices, such as smartphones and tablets, are equipped with two interfaces: a WLAN and a mobile broadband
(e.g., 3G/4G). The operating systems in these devices provide limited support to applications in enabling simultaneous use of
multiple interfaces. There is an initial tussle between power efficiency and performance for manufacturers. The battery costs
have reduced over the period of time providing flexibility to system developers in enabling simultaneous usage of interfaces for 
better performance. However, applications currently exploit only one of these interfaces though there may be an improvement in service
quality to the user with simultaneous use of two interfaces in terms of increased throughput and/or reduced latency. 
To enable such efficient resource usage, multi-path transport protocols are currently being deployed. 
One example is the Multi-path TCP (MPTCP)~\cite{RFC6824} that was recently proposed as an extension to
TCP~\cite{RFC793} to enable the use of multiple paths between two end-hosts for the transmission of a single data stream.

In today's Internet, we observe an increased number of interactive applications that are extremely sensitive to latency, such as web transfers, 
streaming and online gaming. In such applications, the user's experience is affected significantly when data delivery suffers delay. 
Considering the web transfers, the Internet is dominated by web traffic running on top of short-lived TCP connections~\cite{Labovitz-IOR-2009} with
around $95\%$ of client TCP and $70\%$ of server TCP traffic consisting of smaller than ten packets~\cite{Ciullo-IEEECL-2009}.
The quality of user experience when accessing a web page is linked to the download time of the corresponding short-lived flow. As an example,
in~\cite{why-latency-matters-2013}, the authors report that Google measured ``an additional $500$~ms to compute (a web search) [$\ldots$] resulted in a
$25\%$ drop in the number of searches done by users.''

To the best of our knowledge, this is the first research effort that investigates the potential benefits of using multiple paths to reduce latency in order to 
improve user experience. This paper is part of a larger study that assesses the potential latency reduction by leveraging multipath transport protocols. 
While the study discusses and evaluates the performance of different multipath protocols considering various delay sensitive applications, we limit the 
discussion in this paper to the latency performance of MPTCP for web traffic.

The rest of the paper is organizaed as follows: In section~\ref{sec:transports}, we introduce the Multipath TCP protocol and its potential advantage in transporting
web traffic together with the objective of this study. Section~\ref{sec:exp-setup} discusses the experimental setup used for carrying out emulations and
corresponding settings of significance. In section~\ref{sec:exp-results}, we present the results of our emulation experiments together with analysis.
We conclude the paper in section~\ref{sec:conclusion} with brief summary of results and future directions to improve and strengthen our understanding of 
MPTCP latency performance.

