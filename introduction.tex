% ###########################################################################
\section{Introduction}
\label{sec:introduction}
% ###########################################################################
Today, mobile devices such as smartphones and tablets are equipped with two interfaces: a WLAN and a mobile broadband
(e.g., 3G/4G). However, applications currently
exploits only one of these interfaces whereas using both interfaces simultaneously may offer a
better service for the user in terms of increased throughput and/or reduced
latency. To enable such efficient resource usage, multi-path transport protocols
are currently being deployed. One example is the Multi-path TCP
(MPTCP)~\cite{RFC6824} that was recently proposed as an extension to
TCP~\cite{RFC793} to enable the use of multiple paths between two end-hosts
for the transmission of a single data stream.

In today's internet, we observe an increased number of interactive applications
that are extremely sensitive to latency, such as web transfers, streaming and online gaming. In such applications, the user's experience is affected significantly when data delivery suffers delay. To the best of our knowledge, this is the first paper that investigates the potential benefits of using multiple paths to reduce latency in order to improve user experience.

This paper is part of a larger study that assesses the potential latency reduction by leveraging multipath transport protocols. While the study discusses and evaluates the performance of different multipath protocols considering various delay sensitive applications, we limit the discussion in this paper to the latency performance of MPTCP for web traffic.
