% ###########################################################################
\section{Conclusion}
\label{sec:conclusion}
% ###########################################################################

In this paper, we evaluate the latency performance of MPTCP in web traffic with that of TCP using emulations. We observed that MPTCP utilizes the 
available paths efficiently when there is enough data to send and performs better when there is path symmetry. No improvement with MPTCP was observed
when there is path asymmetry and it performs worse when there is not enough data to send. Moreover, Our results indicate that MPTCP provides significant 
gains for websites with large object sizes both for homogeneous and heterogeneous paths. Further studies are planned to understand the congestion response 
of MPTCP in these traffic conditions. 




%The issues encountered by MPTCP are located at various level and local solutions
%may be provided to solve specific problems. In this paper, we list these
%problems to highlight that improving the performance of MPTCP is a complex task.
%We believe that the deployment of MPTCP must be considered in regards with the
%application profile and the very latest versions of single-path TCP.   

