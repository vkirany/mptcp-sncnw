% ###########################################################################
\section{Multi-path TCP for Web Transfer}
\label{sec:transports}
% ###########################################################################

The Internet is dominated by web traffic running on top of short-lived TCP connections~\cite{Labovitz-IOR-2009} with
around $95\%$ of client TCP and $70\%$ of server TCP traffic consisting of smaller than ten packets~\cite{Ciullo-IEEECL-2009}.

The quality of user experience when accessing a web page is linked to the download time of the corresponding short-lived flow. As an example,
in~\cite{why-latency-matters-2013}, the authors report that Google measured ``an additional $500$~ms to compute (a web search) [$\ldots$] resulted in a
$25\%$ drop in the number of searches done by users.''

MPTCP is able to schedule web objects on different paths simultaneously. Therefore, MPTCP can deliver objects on one path even if the other path
experiences loss or excessive buffering. We therefore believe that MPTCP is a good candidate to transport this type of traffic.

%\per{something about why we would expect that MPTCP can help/what properties of
%MPTCP that is potentially beneficial for this type of traffic.}

%\ozgu{also maybe talk about the challenges of multipath, especially in heterogeneous cases, and then conclude we need to investigate and that is what we are doing! Per, I am leaving this to you :) If you like I can have a round in the evening.}

