% ###########################################################################
\section{Multi-path TCP}
\label{sec:transports}
% ###########################################################################
Multi-Path TCP~(MPTCP)~\cite{RFC6824} is a set of of extensions to
TCP~\cite{RFC793,RFC5681} developed by the IETF MPTCP working
group~\cite{MPTCPWG} to enable the simultaneous use of multiple paths between
endpoints. The motivation behind MPTCP is more efficient resource usage and
improved user experience through improved resilience to network failure and
higher throughput.

To use the MPTCP extensions the initiator of a connection appends a
``Multipath Capable''~(MP\_CAPABLE) option in the SYN segment, indicating its support for
MPTCP. When the connection is established, it is possible to add one
TCP flow, or subflow, per available interface to this connection by using a ``MPTCP Join''~(MP\_JOIN)
option in the SYN segment. Once the MPTCP connection has been fully established, both end hosts can
send data over any of the available subflows. While MPTCP transparently
divides user data among the subflows, simultaneous transmission may cause
connection-level packet reordering at the receiver. To handle such reordering,
two levels of sequence numbers are used. Apart from the regular TCP sequence
numbers that are used to ensure in-order delivery at subflow level, MPTCP uses a
64-bit data sequence number that spans the entire MPTCP connection and can be
used to order data arriving at the receiver.

MPTCP also extends the standard TCP congestion control~\cite{RFC2581}, as running existing TCP
congestion control algorithms independently would give MPTCP connections more
than their fair share of the capacity if a bottleneck is shared by two or more of its
subflows. To solve this MPTCP uses a coupled congestion control~\cite{rfc6356}
that links the increase functions of each subflows' congestion control and
dynamically controls the overall aggressiveness of the MPTCP connection. The
coupled congestion control also makes resource usage more efficient as it steers
traffic away from more congested path to less congested paths.

In addition to better resource usage, a goal of MPTCP is to increase
resilience to network failures. By design, the use of multiple paths implicitly
increases resilience as transmission is spread over different paths and the fact
that a path failure can be circumvented by considering other subflows for
transmission. To further improve resilience MPTCP allows senders to retransmit
lost segments on a different subflow. This retransmission strategy enables
moving data from a path that breaks during transmission.

Within the standard operation, MPTCP is able to schedule web objects on different paths simultaneously. 
Therefore, MPTCP can deliver objects on one path even if the other path
experiences loss or excessive buffering. We therefore believe that MPTCP is a good candidate 
to transport this type of traffic. In this study, we aim to understand the latency performance of MPTCP
under various plausible network and traffic scenarios in a systematic manner. In each of the scenario 
we compare the performance with that of TCP to assess the improvement.


